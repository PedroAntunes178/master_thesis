%%%%%%%%%%%%%%%%%%%%%%%%%%%%%%%%%%%%%%%%%%%%%%%%%%%%%%%%%%%%%%%%%%%%%%%%
%                                                                      %
%     File: Thesis_Abstract.tex                                        %
%     Tex Master: Thesis.tex                                           %
%                                                                      %
%     Author: Andre C. Marta                                           %
%     Last modified :  2 Jul 2015                                      %
%                                                                      %
%%%%%%%%%%%%%%%%%%%%%%%%%%%%%%%%%%%%%%%%%%%%%%%%%%%%%%%%%%%%%%%%%%%%%%%%

\section*{Abstract}

% Add entry in the table of contents as section
\addcontentsline{toc}{section}{Abstract}

% Objectives
% Work done
% Conclusions
% up to 250 words

\quad With the advances in new open-source technologies, it's imperial that the new hardware solutions and software implementation on the new hardware is studied. The aim of this thesis is to successfully run a Linux based OS (Operative System) on an IOb-SoC variant. During this work, the implementation of a 32-bit RISC-V CPU capable of running Linux on the Iob-SoC is going to be developed. At the end of this thesis, it's expected to: firstly, be able to run a simulation of the SoC (System on Chip) used to run the Linux kernel and verify its correct functionality; and secondly, implement the IOb-SoC variant developed in an FPGA and successfully boot Linux.

\vfill

\textbf{\Large Keywords:} RISC-V, Linux,
Systems on Chip (SoC), Verilog
