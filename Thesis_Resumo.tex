%%%%%%%%%%%%%%%%%%%%%%%%%%%%%%%%%%%%%%%%%%%%%%%%%%%%%%%%%%%%%%%%%%%%%%%%
%                                                                      %
%     File: Thesis_Resumo.tex                                          %
%     Tex Master: Thesis.tex                                           %
%                                                                      %
%%%%%%%%%%%%%%%%%%%%%%%%%%%%%%%%%%%%%%%%%%%%%%%%%%%%%%%%%%%%%%%%%%%%%%%%

\section*{Resumo}

% Add entry in the table of contents as section
\addcontentsline{toc}{section}{Resumo}

\quad Com o avanço das tecnologias desenvolvidas em código aberto torna-se necessário estudar tanto o novo hardware desenvolvido como o software que tira partido deste novo hardware. Nesta dissertação de mestrado pretende-se conseguir correr um sistema operativo baseado em Linux numa variante do \textit{IOb-SoC}. Ao longo deste trabalho à de-se realizar a implementação de um processador \textit{RISC-V} de \textit{32-bits} capaz de correr o Linux no \textit{IOb-SoC}. No final desta tése pretende-se: em primeiro lugar, ser capaz de correr uma simulação do sistema criado, que mostre o seu correto funcionamento; e em segundo lugar, implementar a variante do IOb-SoC desenvolvida numa FPGA e a partir desta FPGA correr o Linux.

\vfill

\textbf{\Large Palavras-chave:} RISC-V, Linux, Sistema num Chip (SoC), Verilog
