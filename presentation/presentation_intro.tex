%%%%%%%%%%%%%%%%%%%%%%%%%%%%%%%%%%%%%%%%%%%%%%% 
%% presentation_intro.tex

%% Introduction slides for the Thesis Presentation

%% Author: Pedro Miranda
%% Created: 5 Jan 2020
%%%%%%%%%%%%%%%%%%%%%%%%%%%%%%%%%%%%%%%%%%%%%%%


% Content frames
\begin{frame}
  \frametitle{Presentation Outline}
  \begin{itemize}
    %% \item Motivation (use of CGRAs, etc)
  \item Introduction
    \begin{itemize}
    \item Motivation and problem
    \item Proposed solution and thesis objectives
    \end{itemize}
  \item Hardware platform (IOb-SoC)
    \begin{itemize}
      \item Integrate additional peripherals
      %% \item Establish a testing environment
      \item Profile Tiny YOLOv3 application
    \end{itemize}
  \item VersatCNN acceleration
    \begin{itemize}
    \item VersatCNN dataflow configuration
    \item Tiny YOLOv3 acceleration strategies
      \begin{itemize}
      \item CNN inference
      \item Pre and post-processing
      \end{itemize}
    \end{itemize}
  \item Results and Conclusions
    \begin{itemize}
      \item Comparison with other platforms and works
      \item Achievements and future improvements
    \end{itemize}
  \end{itemize}
\end{frame}
%% This slide presents outlines the work and the presentation structure.
%% Is used to give the audience a road map for the presentation.

%% The work requires the establishment of a testing environment. For that we use
%% the IOb-SoC as an hardware platform to which we add more peripherals.
%% At the end, the Tiny YOLOv3 is profiled running of the baseline system to
%% identify the target functions to accelerate.

%% After that, the VersatCNN is added as a system peripheral. The Tiny YOLOv3
%% acceleration requires different strategies, which are reflected in different
%% datapath configurations.

%% The presentation concludes with a discussion of the obtained results and a
%% comparison with other Tiny YOLOv3 implementations for FPGAs. 


\begin{frame}
  \frametitle{Motivation}
  \begin{itemize}
  \item Object detection is usefull in multiple areas of application
    (navigation, medical, security)
    \begin{itemize}
    \item Most accurate methods use CNN inference.
      \item \textbf{Problem:} Computationally demanding process.
    \end{itemize}
  \item Current solutions based on GPU acceleration.
    \begin{itemize}
      \item Oversized and power-hungry.
    \end{itemize}
  \item Alternative accelerators also have limitations:
    \begin{itemize}
    \item FPGA: better size/power efficiency than GPU, with configuration
      overhead and increased development time.
    \item ASIC: dedicated accelerators lack hardware programmability.
    \end{itemize}
  \item \textbf{Proposed solution:} CGRA
    \begin{itemize}
    \item Optimized area and power.
    \item Faster reconfiguration for FPGA implementation.
      \item Adds hardware programmability for ASIC implementation.
    \end{itemize}
  \end{itemize}
\end{frame}

% Thesis Objectives
\begin{frame}
  \frametitle{Thesis Goals}
  Main thesis goals:
  \begin{itemize}
  \item Accelerate a Convolutional Neural Network (CNN) for object detection,
    using the VersatCNN CGRA 
    \begin{itemize}
      %% \item Chosen application: Tiny YOLOv3
    \item Configure CGRA dataflow to efficiently implement the Tiny YOLOv3
      application
    %% \item Design CGRA dataflow configurations to accelerate Tiny YOLOv3 kernels
    \end{itemize}
  \item Target real-time processing (30 FPS)
  \end{itemize}
\end{frame}
%% The main goal of this Thesis is to use the VersatCNN CGRA to accelerate a CNN
%% application.

%% Tiny YOLOv3 is the CNN chosen.

%% The project has real-time (30 FPS) target performance.

